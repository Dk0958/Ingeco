%----------------------------------------------------------------------------------------
%	Ejercicios
%----------------------------------------------------------------------------------------

\chapterimage{A_Capitulo_Ejercicios/img/portada/ima2} % Chapter heading image

\chapter {Ejercicios}

\addcontentsline{toc}{chapter}{\textcolor{ocre}{Capítulo 1}}


\begin{itemize}
 \item 1. Calcular el interés simple 300.000 COP desde el 18 de marzo al 18 de junio del mismo año al 3,4\% periodo mes vencido.\\
       \textbf{Respuesta:} COP 30.600
       \medskip

 \item 2. Una persona invierte COP 250.000 al 40\% "nominal anual año vencido" desde el 15 de septiembre de 1998 hasta el 18 de noviembre de 1998. Calcular:
       \\
       a) el valor final racional y
       b) el valor final bancario.
       \\
       \textbf{Respuestas:} a) COP 267.534,25  b) COP 2'677.777,78
       \medskip

 \item 3. ¿Cuánto debe invertirse hoy 17 de octubre en un fondo que garantiza el 28\% "nominal anual año vencido" para que el 20 de marzo del siguiente año pueda retirar la suma de COP 150.000?\\
       \textbf{Respuesta:} COP 134.151,72
       \medskip


 \item 4. Hallar el valor presente de COP 500.000 en 3 y 1/2 años al 3\%período mes vencido\\
       \textbf{Respuesta:} COP 221.239
       \medskip

 \item 5. Hace 6 años compré un lote en COP 900.000 y hoy se vendió en COP 6 millones. Hallar la tasa de interés "nominal anual año vencido" comercial que gané en este negocio.\\
       \textbf{Respuesta: }94,44\% "nominal anual año vencido"
       \medskip

 \item 6. ¿Qué tan rentable es un documento que hoy se puede comprar en COP 750.000 el cual devolverá al cabo de 3 años la suma de COP 3'300.000?\\
       \medskip

 \item 7. Se recibe un préstamo por  COP  1 millón al 42\% nominal anual año vencido el día 8 de agosto de 1999 con vencimiento el 8 de marzo del 2000. Hallar el valor final del préstamo calculando los intereses (I):\\

       a) interés exacto o racional\\
       b) interés comercial o base 360\\
       e) interés bancario\\
       d) interés base 365\\
       Tenga en cuenta que el año 2000 es un año bisiesto\\
       \textbf{ Respuestas:} a) COP 1'244.426,23; b) COP 1'245.000; e) COP 1'248.500; d) COP 1'243.945,21
       \medskip

 \item 8. Un pagaré con valor presente de  COP  300.000 emitido el 15 de septiembre de 1999 con plazo de 270 días a una tasa de interés del 30\% nominal anual año vencido.\\
       Hallar el valor futuro y la fecha de vencimiento en:\\

       a) interés exacto o racional\\
       b) interés comercial o base 360\\
       c) interés bancario\\
       d) interés base 365\\
       \textbf{Respuestas:} a) COP 366.393,44; 11-06-00; b) COP 367.500; 15-06-00; c) COP 367.500; 11-06-00; d) COP 366.575,34; 12-06-00
       \medskip

 \item 9. Una letra por COP 550.000 madura el 23 de agosto de 1998 y va a ser descontada el 17 de julio del mismo año al 38\% "nominal anual año vencido". Determinar el valor de la transacción.\\
       \textbf{Respuesta:} COP 528.519,44
       \medskip

 \item 10. El 15 de diciembre de 1999 una empresa recibe un pagaré por  COP  2 millones a un plazo de 3 meses al 25\% nominal anual año vencido de interés comercial simple. El 14 de enero lo negocia con un banco que lo adquiere a una tasa de descuento del 29\% nominal anual año anticipado en interés bancario. ¿Cuánto recibirá la empresa por el pagaré y cuánto ganará el banco en la operación de descuento?\\

       \textbf{Respuestas:} la empresa recibe COP 2.020.579,86; el banco gana COP 104.420,14
       \medskip

 \item 11. Halle el valor de maduración de un pagaré con vencimiento el 20 de abril si va a ser descontado el 13 de marzo del mismo año al 40\% "nominal anual año vencido" y su valor de transacción es de  COP  780.400\\
       \medskip

 \item 12. Una persona solicita un préstamo a un banco por la suma de COP 800.000, a un plazo de 90 días y le cobran una tasa anticipada del 38\% "nominal anual año vencido".\\

       a) ¿Cuál es el valor líquido que le entregan?\\
       b) Suponga que el banco cobra COP 15 000 por el estudio del crédito, ¿cuál será el valor liquido?\\

       \textbf{Respuestas:} a) COP 724.000; b) COP 709.000
       \medskip

 \item 13. ¿Cuál es el valor del documento que queda en poder de un banco, si el prestatario recibe un valor liquido de  COP 500.000 por un documento con maduración en 90 días, si le cobran una tasa de descuento del 41\% "nominal anual año vencido"? \\

       a) Sin tener en cuenta costos de apertura del crédito y\\
       b) Teniendo en cuenta que el banco cobra COP 2 000 por estudio del documento \\
       \medskip

 \item 14. Un documento de valor inicial COP 700.000 es fechado el 25 de septiembre de 1998 a un plazo de 325 días y un interés del 32\% "nominal anual año vencido". Si es descontado por un banco el 18 de marzo de 1999 al 40\% "nominal anual año vencido" determinar:\\

       a) La fecha de vencimiento\\
       b) El valor al vencimiento\\
       c) El valor de transacción \\
       Usar interés bancario\\

       \medskip

 \item 15. Hallar la verdadera tasa bancaria que cobra un banco cuando descuenta un documento con valor de maduración de COP 400.000 si es descontado 25 días antes del vencimiento al 41\% nominal anual año anticipado.\\

       \textbf{Respuesta:} 42,2\% nominal anual año vencido
       \medskip

 \item 16. Un almacén ofrece los siguientes descuentos, sobre una mercancía cuyo costo inicial es de COP 200.000:\\

       30\% por venta al por mayor, 10\% por pago al contado y 5\% por enviar la mercancía sin empaque.\\

       a)¿Cuál es el valor final de la factura?\\
       b) ¿Cuál es el descuento promedio que se otorgó?\\

       \textbf{Respuestas:} a) COP 119.700 b) 40,15\%
       \medskip

 \item 17. Una fábrica ofrece un descuento del 25\% en ventas al por mayor, el 5\% por pronto pago y el 4\% por embalaje. ¿Cuál debe ser el descuento adicional que puede ofrecerse a los empleados de la misma fábrica para que el descuento total no sea superior al 35\%?\\
       \textbf{Respuesta:} 4,971%\
       \medskip

 \item 18. Se compra un artículo por COP 870.000 el día 25 de noviembre y se acuerda que será cancelado mediante el sistema de pagos parciales, con un plazo máximo de 3 meses. Si el día de la compra se da una cuota inicial del 30\%, el 12 de diciembre se hace un abono parcial de COP 200.000 y el 20 de enero del siguiente año se hace otro abono parcial de COP 150.000 ¿cuál debe ser el valor del pago final que hecho al vencimiento cancelará la deuda? Suponga que se carga un interés bancario del 34\% "nominal anual año vencido".\\

       \textbf{Respuesta:} COP 293.865,71
       \medskip

\end{itemize}