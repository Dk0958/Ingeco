\chapter*{Capítulo 5}
\addcontentsline{toc}{chapter}{\textcolor{ocre}{Capítulo 5}}



\begin{itemize}

 \item 1. Cuando su hijo cumple 12 años, un padre hace un depósito de "X" COP en una fiduciaria, con el objeto de asegurar sus estudios universitarios, los cuales iniciará cuando cumpla 20 años. Suponiendo que para esa época el valor de la matrícula anual en la universidad será de \$300.000 COP y que permanecerá constante durante los seis años que duran los estudios universitarios, ¿cuál debe ser el valor de "X" COP? Suponga una tasa del 30\% naav (nominal anual año vencido"), anual efectiva.\\
       \textbf{Respuesta:} El valor del depósito en la fiduciaria debe ser igual a \$126.349,41 COP\\
       \medskip

 \item 2. Una persona necesita comprar hoy una máquina, el modelo A cuesta \$300.000 COP; el modelo B, \$500.000 COP, el \$700.000 COP y el modelo D, \$900.000 COP. Si la persona puede hacer 36 pagos mensuales de máximo \$30.000 COP durante 3 años, pero comenzando el primer pago al final de 6 meses ¿cuál será el modelo más costoso que podrá comprar? Suponga una tasa del 30\% namv (nominal anual mes vencido") .\\
       \textbf{Respuesta:} El valor presente que puede pagar es \$624.608,81 COP; esto indica que el modelo más costoso que puede comprarse es el modelo B.\\
       \medskip

 \item 3. Una persona solicita un préstamo el día 1 de marzo de 1989 y planea efectuar pagos mensuales de \$120.000 COP, desde el 1 de octubre de 1989, hasta el 1 de agosto de 1990. Si le cobran un interés del 3.5\% período mes vencido , ¿cuál será el valor del préstamo?\\
      \textbf{Respuesta:} El valor del préstamo será de \$87.873,21 COP.\\
       \medskip

 \item 4. Un filantrópico ha creado una institución de caridad y desea asegurar su funcionamiento a perpetuidad. Se estima que esta institución necesita para su funcionamiento \$100.000 COP, al final de cada mes, durante el primer año; \$200.000 COP, al final de cada mes, durante el segundo año y \$300.000 COP, al final de cada mes, en forma indefinida. Suponiendo una tasa del 30\% namv (nominal anual mes vencido"), hallar el valor de la dote (Se denomina dote al valor único que, entregado al principio, asegura el mantenimiento a perpetuidad, en el caso de las personas el mantenimiento es vitalicio).\\
       \textbf{Respuesta:} El valor de la dote será de \$9.185,725 COP\\
       \medskip

 \item 5. Un señor deposita el primero de abril de 1986 \$100.000 COP, en un fondo que paga el 36\% nasv (nominal anual semestre vencido").\\

       a. ¿cuántos retiros semestrales de \$8.000 COP podrá hacer, si el primer retiro lo hace el primero de abril de 1989?\\
       b. ¿cuál será el valor del retiro adicional que hecho un período después del último pago de \$8.000 COP cancelará el fondo?\\
       \textbf{Respuesta:} Si el primer retiro lo hace el primero de abril de 1989 podrá hacer 4 retiros semestrales (psv) de \$8.000 COP\\
       El valor del retiro adicional es de \$3104,68 COP\\
       \medskip

 \item 6. Suponiendo una tasa del 36\% namv (nominal anual mes vencido"), ¿cuál será el valor presente de:\\

       a.\$200.000 COP, al final de cada mes, en forma indefinida\\
       b.\$200.000 COP, al principio de cada mes indefinidamente?\\
       \textbf{Respuestas:} El valor será de \$6.666.666,67 COP al final de cada mes de forma indefinida\\
       El valor será de \$6.866.666,67 COP al principio de cada mes de forma indefinida\\
       \medskip

 \item 7. Un inversionista deposita hoy \$100.000 COP y \$300.000 COP, en 3 años; al final del año 5, comienza a hacer depósitos anuales de \$50.000 COP, durante 6 años, ¿cuánto dinero podrá retirarse en forma indefinida, comenzado al final del año 14? Utilice una tasa del 20\% naav (nominal anual año vencido").\\
       \textbf{Respuesta:} Se podrá retirar de forma indefinida \$757.080 COP\\
       \medskip

 \item 8. Un grupo de benefactores decide donar a un hospital de los equipos de laboratorio que necesita, se estima que el costo de los equipos el día primero de julio de 1990 será de COP 4 millones y que necesitará \$300.000 COP trimestralmente, como costo de funcionamiento en forma indefinida, a partir del primero de abril de 1991, fecha en la cual entrará en funcionamiento. ¿Cuál debe ser el valor de la donación que se haga el día primero de enero de 1990 si el dinero es invertido inmediatamente en una fiduciaria que garantiza el 24\%  natv (nominal anual trimestre vencido") ?\\
       \textbf{Respuesta:} El valor de la donación debe ser de \$7’520.454 COP\\
       \medskip

 \item 9 Una empresa pretende tomar una casa-lote que requiere la suma de \$2.000.000 COP anuales como costo de mantenimiento y de \$3.000.000 COP cada 4 años para reparaciones adicionales. Por otra casa-lote que le ofrecen, se requerirá de una suma de \$3.000.000 COP anuales para mantenimiento y de \$2.500.000 COP cada tres años para reparaciones adicionales. Si la casa-lote se usará por tiempo indefinido y suponiendo una tasa de interés del 35\% naav (nominal anual año vencido"), ¿cuál de las dos alternativas es más conveniente tomar?\\
       \textbf{Respuesta:} Es más conveniente elegir la primera alternativa de \$7.006.550 COP\\
       \medskip

 \item 10. Con interés al 24\%  natv (nominal anual trimestre vencido"), ¿cuál debe ser valor de los pagos semestrales vencidos que, hechos por 10 años, que amortizarán una deuda de \$1.200.000 COP?\\
       \textbf{Respuesta:} El valor de los pagos debe ser de \$164.293 COP\\
       \medskip

 \item 11. Resolver el problema anterior si los pagos son anticipados.\\
       \textbf{Respuesta:} Teniendo en cuenta los valores anticipados el valor de los pagos debe ser de \$146.220 COP\\
       \medskip

 \item 12. Una persona compra un artículo por \$6.000.000 COP. Sí da una cuota inicial del 40\% y cancela el saldo, en cuotas trimestrales vencidas, de \$500.000, cancelará la deuda.\\
      \textbf{Respuesta:} Deberán hacerse 12 pagos de \$50.000 COP y un pago final de \$21.631 COP\\
       \medskip

 \item 13. Si se deposita mensualmente la suma de \$100.000 COP en un fondo que paga el 27\%  namv (nominal anual mes vencido")  y adicionalmente, deposita de \$200.000 COP cada 3 meses ¿Cuánto se habrá acumulado, al final de 5 años?\\
      \textbf{Respuesta:} Se habrá acumulado un total de \$205.578 COP\\
       \medskip

 \item 14. Con  el objeto de poder hacer 10 retiros semestrales de \$700.000 COP, se depositan hoy un capital en una cuenta de ahorros que paga el 21\%  natv (nominal anual trimestre vencido"). Si el primer retiro lo hace al final de un año, ¿cuál debe ser el valor de la cuenta?\\
      \textbf{Respuesta:} El valor de la cuenta debe ser de \$375.673 COP\\
       \medskip

 \item 15. Un señor desea comprar una póliza de seguro que garantice a su esposa el pago de \$400.000 COP mensuales durante 10 años y adicionalmente \$500.000 COP al final de cada año durante este mismo período. Si el primer pago se efectúa al mes del fallecimiento del señor, hallar el valor de la póliza de seguro suponiendo que la compañía de seguros garantiza el 24\%  namv (nominal anual mes vencido").\\
      \textbf{Respuesta:} El valor de la poliza es de \$1.983.299,51 COP\\
       \medskip

 \item 16. Se desea cancelar una deuda de \$900.000 COP en pagos mensuales de "R" COP durante 3 años, el primero al final de un mes, y además se efectuaran abonos semestrales extraordinarios de una y media veces la cuota ordinaria, el primero de estos al final de 6 meses. Suponiendo una tasa del 36\% namv (nominal anual mes vencido"). ¿Cuál debe ser el valor de las cuotas ordinarias y el de las cuotas extraordinarias?\\
       \textbf{Respuesta:} El valor de las cuotas ordinarias es de \$33.463,38 COP\\ 
       El valor de las cuotas extraordinarias es de \$50.195,07 COP\\
       \medskip

 \item 17. Se compra un carro en \$12.000.000 COP mediante el pago de 48 cuotas mensuales vencidas de  COP "R" c/u y cuotas trimestrales vencidas de \$400.000 COP c/u durante 4 años. Si se cobra una tasa del 44\% naav (nominal anual año vencido"), determinar el valor de  COP "R".\\
       \textbf{Respuesta:} El valor de R es de \$353.137,09 COP\\
       \medskip

 \item 18. Con una tasa del 25\% naav (nominal anual año vencido"), ¿cuál debe ser el valor presente de una serie uniforme infinita de \$600.000 COP al final de cada 4 años? Utilice cambio de tasa.\\
       \textbf{Respuesta:} El valor presente de la serie es \$416.260 COP\\
       \medskip

 \item 19. Resuelva el problema anterior modificando los pagos\\
       \textbf{Respuesta:} Al modificar los pagos el valor seguiría siendo \$416.260 COP\\
       \medskip

 \item 20. Reemplazar pagos de \$200.000 COP hechos cada 2 años por pagos equivalentes cada 5 años suponiendo una tasa del 30\% naav (nominal anual año vencido")\\
       \textbf{Respuesta:} El equivalente serian pagos de \$786.356,52 COP.\\
       \medskip

 \item 21. Con una tasa del 20\% naav (nominal anual año vencido"), ¿qué es más conveniente para una universidad, recibir una renta perpetua de \$800.000 COP cada 5 años comenzando el primer pago en el cuarto año, 0 recibir \$200.000 COP anuales de renta perpetua comenzando el primero dentro de un año?\\
       \textbf{Respuesta:} La mejor opción es la segunda correspondiente a \$1.000.000 COP\\
       \medskip

 \item 22. Una máquina llegará al final de su vida útil dentro de 2 años, para esa época una nueva máquina que se adquiera costará \$900.000 COP y se estima que la máquina vieja podrá ser recibida en parte de pago de la nueva en la suma de \$200.000 COP. ¿Qué depósito trimestral debo hacer en una cuenta que paga el 30\% namv (nominal anual mes vencido"), con el objeto de hacer la compra en el momento oportuno si el primer depósito lo hago al final de 6 meses?\\
       \textbf{Respuesta:} Debería hacer un depósito de \$79.200,82 COP\\
       \medskip

 \item 23. Una fábrica se puede comprar en la suma de 1 millón COP, la cual produce 2.000 unidades mensuales de un cierto artículo que podrá ser vendido durante los primeros 6 meses a 30 COP la unidad y en 50 COP la unidad durante los siguientes 6 meses. El inversionista piensa que podrá vender la fábrica al final de un año en la suma de 1.2 millones COP. Si el inversionista gana normalmente en todos sus negocios el 5\% período mes vencido, le aconsejaría usted que comprara la fábrica?\\
       \textbf{Respuesta:} Si le aconsejaría comprar la fábrica ya que los ingresos son de \$1.351.502,38 COP y los egresos son de \$1.000.000 COP\\
       \medskip

 \item 24. Calcular la tasa que gana el inversionista del problema anterior.\\
       \textbf{Respuesta}: La tasa que gana el inversionista es de 8.54\% pmv (periodo mes vencido)\\
       \medskip

 \item 25. Hoy primero de noviembre de 1999 se tiene una obligación a la que le restan 18 cuotas mensuales anticipadas de \$827.643 COP c/u para terminar de pagarse. El acreedor desea cambiar la forma de pago de su deuda y pacta realizar 10 pagos trimestrales vencidos de \$2.965.345.96 COP c/u. ¿En qué fecha deberá realizar el primer pago de la nueva anualidad? Utilice una tasa del 38.5\% naav (nominal anual año vencido").\\
       \textbf{Respuesta:} Debe realizar el primer pago correspondiente al periodo 7, que sería el primero de agosto de 2001.\\
       \medskip

 \item 26. Se concede un préstamo de 20 millones COP para ser pagado en cuotas mensuales en la siguientes condiciones:\\
       a. En los primeros 6 meses no se efectuara ningún pago pero los intereses si se causan\\
       b. En los siguientes 6 meses se pagará la suma de COP600.000 mensualmente\\
       c. De el mes 13 en adelante se pagará mensualmente la suma de  COP  "R"\\
       d. Plazo total 3 años.\\
       e. Durante el primer año se cobrará un interés del 12\% namv (nominal anual mes vencido"), durante el segundo año se cobrara una tasa del 18\%  namv (nominal anual mes vencido"), y durante el tercer año se cobrará una tasa del 24\%  namv (nominal anual mes vencido").\\
       Determinar el valor de  COP  "R"\\
       \textbf{Respuesta:} El valor de R es de \$954.066,98 COP\\
       \medskip

 \item 27. Por un préstamo de 15 millones se exige el pago de \$450.000 COP mensuales durante los primeros 12 meses, de \$600.000 COP por los siguientes 12 meses, de 1 millón mensual COP por los siguientes 12 meses y un pago final de COP 2 millones en el mes 3. ¿Cuál es la tasa de interés periódica mensual que se está cobrando?\\
       \textbf{Respuesta:} Se está cobrando una tasa periódica de 2,725\% pmv (periodo mes vencido)\\
       \medskip

 \item 28. Resolver el problema anterior suponiendo que a partir del segundo año se cobra una tasa de interés igual al doble de la que se cobra en el primer año.\\
       \textbf{Sugerencia:} Resuelva el problema por Excel.\\
       \textbf{Respuesta:} La tasa periódica del primer año es de 1,4356\% pmv (periodo mes vencido) y ahí en adelante es de 2,8712\% pmv (periodo mes vencido)\\
       \medskip

\end{itemize}