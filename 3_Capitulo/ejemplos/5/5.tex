\newpage
\textbf{Ejemplo 5}\\
Una persona tiene un préstamo hipotecario al IPC + 4 puntos ¿Cuál
debe ser el spread si se cambia a otro plan cuya tasa es la DTF +X?
Suponga que el IPC = 8\% efectiva anual y que la DTF = 18,67\% nominal
anual trimestre anticipado, en donde IPC: Tasa del Índice de Precios al
Consumidor DTF: Tasa de Depósito a Término Fijo promedio.\\ \\
%\newpage %USAR SOLO SI EL SOLUCIÓN QUEDA SOLO Y ES NECESARIO BAJARLO A LA SIGUIENTE PAGINA
\textbf{Solución.}\\
%La tabla ira centrada
\begin{center}
 \renewcommand{\arraystretch}{1.5}% Margenes de las celdas
 %Creación de la cuadricula de 3 columnas
 \begin{longtable}[H]{|p{0.3750\linewidth}|p{0.3750\linewidth}|p{0.25\linewidth}|}
  %Creamos una linea horizontal
  \hline
  %Definimos el color de la primera fila
  \rowcolor[HTML]{FFB183}
  %%%%% INICIO ASIGNACIÓN FECHA FOCAL %%%%%%%
  %%%%%%%%%% INICIO TITULO
  %Lo que se hace aquí es mezclar las 3 columnas en una sola
  \multicolumn{3}{|c|}{\cellcolor[HTML]{FFB183}\textbf{1. Asignación período focal}}                                                                \\ \hline
  %%%%%%%%%% FIN TITULO
  \multicolumn{3}{|c|}{  $pf = 1 \textit{ pav} = 4 \textit{ ptv}$}                                                                                \\ \hline

  %%%%% INICIO DECLARACIÓN DE VARIABLES %%%%%%%
  %%%%%%%%%% INICIO TITULO
  %Lo que se hace aquí es mezclar las 3 columnas en una sola
  \multicolumn{3}{|c|}{\cellcolor[HTML]{FFB183}\textbf{2. Declaración de variables}}                                                              \\ \hline
  %%%%%%%%%% FIN TITULO
  %%%%%%%%%% INICIO DE MATEMÁTICAS
  %Cada & hace referencia al paso de la siguiente columna
  $j_{1} = 8\% \textit{ naav}$     & $m_{1} = 1 \textit{ naav} $ & $X = ? \% $                                                                       \\
  $j_{2} = 18,67\% \textit{ nata}$ & $m_{2} = 4 \textit{ ptv} $  &                                                                                \\ &                             & \\ & 				           					 & 	 \\ \hline

  %%%%%%%%%% FIN DE MATEMÁTICAS
  %%%%% FIN DECLARACIÓN DE VARIABLES


  %%%%% INICIO DECLARACIÓN FORMULAS
  %%%%%%%%%%% INICIO TITULO
  \rowcolor[HTML]{FFB183}
  \multicolumn{3}{|c|}{\cellcolor[HTML]{FFB183}\textbf{3. Declaración de fórmulas}}                                                               \\ \hline
  %%%%%%%%%%% FIN TITULO
  %%%%%%%%%%% INICIO MATEMÁTICAS
  \multicolumn{3}{|l|}{$IPC + 4 = DTF + X \hspace{1cm}\textit{Ecuación de valor}$}                                                                \\
  \multicolumn{3}{|l|}{$i = i_{1} + i_{2} + (i_{1})(i_{2})\hspace{1cm}\textit{Tasas combinadas}$}                                                 \\

  \multicolumn{3}{|l|}{$i_{a} =i_{1} + i \hspace{3cm}\textit{Tasa de interés períodica anticipada}$}                                              \\
  \multicolumn{3}{|l|}{$ (1 + i_{1})\cdot m_{1} = (1 + i2)\cdot m_{2} \hspace{1cm}\textit{Equivalencia tasas anticipadas}$}                       \\
  \multicolumn{3}{|l|}{$ j_{a} = i_{m} \hspace{1cm}\textit{Tasa nominal anual}$}                                                                  \\
  \multicolumn{3}{|l|}{$\textit{Tasa del crédito} = IPC + 4 \textit{ puntos efectivos anuales} = DTF + x( \textit{ spread en}\% \textit{ nata})$} \\ \hline

  %%%%%%%%%% FIN MATEMÁTICAS
  %%%%%% INICIO DESARROLLO MATEMÁTICO
  \rowcolor[HTML]{FFB183}
  %%%%%%%%%%INICIO TITULO
  \multicolumn{3}{|c|}{\cellcolor[HTML]{FFB183}\textbf{5. Desarrollo matemático}}                                                                 \\ \hline
  %%%%%%%%%% FIN TITULO
  %%%%%%%%%% INICIO MATEMÁTICAS
  \multicolumn{3}{|p{\textwidth}|}{
  Calcularemos la tasa neta del crédito inicial que está en naav, a partir de la fórmula de tasas
  combinadas, las fórmulas interés periódica vencida y luego la tasa periódica trimestre
  anticipada, para luego igualar la tasa nominal anual trimestre anticipada y despejar la X=?
  nata.\newline
  \setlength{\parskip}{0.1mm}

  $IPC + 4 = 0,08 + 0,04 + 0,08 \cdot 0,04$\newline

  $\textit{Tasa del crédito} = IPC + 4 \textit{ puntos}$\newline

  $\textit{Tasa de crédito inicial} = 0,08 + 0,04 + 0,08 \cdot 0,04 \textit{ naav}$\newline

  $IPC + 4 = 12,32\%\textit{ pav equivalente a} 12,32\%\textit{ naav}$\newline

  $(1 + 0,1232)^1 = (1 + i)^4$\newline

  $i = 2,947137112\%\textit{ ptv}$\newline

  $i_{a} =\frac{0,02947137112}{1 + 0,02947137112} = 0,0286275\%\textit{ pta}$\newline

  $ja = 2,86875  \cdot  4 = 11,451\%\textit{ nata}$\newline

  $\textit{Tasa de crédito equivalente} = DTF + X(\%\textit{ nata})$\newline

  $\textit{Tasa de crédito equivalente} = 11,45\% \textit{ nata} = 18,67\% \textit{ nata} + X\%\textit{ nata}$\newline

  $\textit{Entonces } X = 11,451\% \textit{ nata} - 12,32\% \textit{ nata} = -7,219\%nata$\newline

  $DTF + X = 0,1867 + X \textit{ nata}$\newline

  $0,11451 = 0,1867 + \textit{ nata donde X}   = -0,07219$\newline

  $X = -7,219\% \textit{ nata}$\newline
  \setlength{\parskip}{0mm}
  }                                                                                                                                               \\ \hline


  %%%%%%%%%% FIN MATEMÁTICAS
  %%%%%% FIN DESARROLLO MATEMÁTICO
  %%%%%% INICIO RESPUESTA
  \rowcolor[HTML]{FFB183}
  %%%%%%%%%%INICIO TITULO
  \multicolumn{3}{|c|}{\cellcolor[HTML]{FFB183}\textbf{6. Respuesta}}                                                                             \\ \hline
  %%%%%%%%%% FIN TITULO
  %%%%%%%%%% INICIO RESPUESTA MATEMÁTICA
  \multicolumn{3}{|C{\textwidth}|}{
  $X = -7,219\% \textit{nata}$
  }                                                                                                                                               \\ \hline


  %%%%%%%%%% FIN MATEMÁTICAS
  %%%%%% FIN RESPUESTA
 \end{longtable}
 %Se crean dos lineas en blanco para que no quede el siguiente texto tan pegado
 %\newline \newline %USARLO SI CREES QUE ES NECESARIO
\end{center}