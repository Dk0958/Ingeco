\newpage
\textbf{Ejemplo 9}\\
El dueño del almacén solicita al banco una aceptación bancaria por 5 millones COP, que
entregará al fabricante o proveedor para que le entreguen la mercancía. El banco presta
este servicio y le cobra al almacén una comisión del 1\% mensual trimestre anticipado sobre
el valor de la aceptación. Considerar el iva del 16\% del valor monetario de la comisión
bancaria.\\
a) Comisión bancaria que debe pagar el almacén por su expedición.\\
Si el fabricante vende la aceptación faltando 40 días a una tasa del 30\% pdv y por ello el
precio de registro en bolsa es de $P_{R}$ = 4.850.000 COP, y la bolsa le hace una retención en la fuente sobre los rendimientos del 7\% calcular: \\
b) El precio de compra Pc incluyendo la retención en la fuente y excluyendo la comisión
del comisionista comprador. \\ 
\textbf{Solución.}
%La tabla ira centrada
\begin{center}
	\renewcommand{\arraystretch}{1.5}% Margenes de las celdas
	%Creación de la cuadricula de 3 columnas
\begin{longtable}[H]{|c|c|c|}
		%Creamos una linea horizontal
\hline
		%Definimos el color de la primera fila
\rowcolor[HTML]{FFB183}
		%%%%% INICIO ASIGNACIÓN PERIODO FOCAL %%%%%%%
		%%%%%%%%%% INICIO TITULO
		%Lo que se hace aquí es mezclar las 3 columnas en una sola
\multicolumn{3}{|c|}{\cellcolor[HTML]{FFB183}\textbf{1. Asignación período focal}}   \\ \hline
\multicolumn{3}{|c|} {$pf = 40 pdv$} \\ \hline
		%%%%%%%%%% FIN TITULO
  %%%%% INICIO DECLARACIÓN FORMULAS
  
%%%%%%%%%%% INICIO TITULO
\rowcolor[HTML]{FFB183}
\multicolumn{3}{|c|}{\cellcolor[HTML]{FFB183}\textbf{2. Declaración de variables}}    \\ \hline
%%%%%%%%%%% FIN TITULO
%%%%%%%%%%% INICIO MATEMÁTICAS
\begin{minipage}{9cm}
\textit{Com bancaria = 1\% mensual trimestre anticipado del valor de la aceptación }   \end{minipage}      &  \multicolumn{2}{c|}{$ V_{com} = \textit{? COP Valor comisión bancaria} $}
\\ 
\begin{minipage}{9cm}
\textit{iva = 16\% de la com bancaria\\
$n = \frac{40}{365} p(40 dv)$\\
P\textsubscript{R} = 4.858.285 COP\\
R\textsubscript{F} = 7\% sobre las utilidades (F - P\textsubscript{R})\\}   \end{minipage}      & \multicolumn{2}{c|}{$ P_{c} = \textit{? COP} $}
\\ \hline
%%%%%%%%%% FIN MATEMÁTICAS
		%%%%% INICIO FLUJO DE CAJA
\rowcolor[HTML]{FFB183}
\multicolumn{3}{|c|}{\cellcolor[HTML]{FFB183}\textbf{3. Diagrama de flujo de caja}} \\ \hline
		%Mezclamos 3 columnas y pondremos el dibujo
		%%%%%%%%%%%%% INSERCIÓN DE LA IMAGEN
		%Deberán descargar las imágenes respectivas del drive y pegarlas en la carpeta
		%n_capitulo/img/ejemplos/1/capitulo1ejemplo1.pdf  (el /1/ es el numero del ejemplo)
\multicolumn{3}{|p{\textwidth}|}{No aplica porque el enunciado ya entrega el precio de registro y el objetivo del ejercicio es
aplicar los conceptos de expedición, aceptaciones bancarias, valor del iva y de retención
en la fuente.}  \\ \hline
		%%%%%%%%%%%%% FIN INSERCIÓN DE IMAGEN
		%%%%%FIN FLUJO DE CAJA
		
		
		
		%%%%% INICIO DECLARACIÓN FORMULAS
		%%%%%%%%%%% INICIO TITULO
\rowcolor[HTML]{FFB183}
\multicolumn{3}{|c|}{\cellcolor[HTML]{FFB183}\textbf{4. Declaración de fórmulas}}    \\ \hline
		%%%%%%%%%%% FIN TITULO
		%%%%%%%%%%% INICIO MATEMÁTICAS
\multicolumn{3}{|c|}{\textit{a) Com bancaria = F(\% que cobre el banco + iva)}} \\
\multicolumn{3}{|c|}{\textit{iva = 16\% $\cdot$ Com bancaria}} \\
\multicolumn{3}{|c|}{\textit{b) $P_{c} = P_{R} + R_{F}$}} \\
\multicolumn{3}{|c|}{\textit{$R_{F} = 7\%(F-P_{R})$}} \\
\hline	
	
		%%%%%%%%%% FIN MATEMÁTICAS
		%%%%%% INICIO DESARROLLO MATEMÁTICO
\rowcolor[HTML]{FFB183}
		%%%%%%%%%%INICIO TITULO
\multicolumn{3}{|c|}{\cellcolor[HTML]{FFB183}\textbf{5. Desarrollo matemático}}       \\ \hline
		%%%%%%%%%% FIN TITULO
		%%%%%%%%%% INICIO MATEMÁTICAS

\multicolumn{3}{|p{\textwidth}|}{\textit{a) Com bancaria = 5.000.000 $COP\cdot 0,01$ (3 meses) = 150.000 COP} }
\\
\multicolumn{3}{|p{\textwidth}|}{\textit{iva = 150.000 COP(0,16) = 24.000 COP} } \\
\multicolumn{3}{|p{\textwidth}|}{\textit{Com bancaria + iva = 150.00 COP + 24.000 COP = 174.000 COP que afecta al almacén} } \\
\multicolumn{3}{|p{\textwidth}|}{\textit{b) $R_{F}$ = 0,07 (5.000.000 COP - 4.858.285 COP) = 9.920 COP} }
\\ \hline
		
\hline
		
		%%%%%%%%%% FIN MATEMÁTICAS
		%%%%%% FIN DESARROLLO MATEMÁTICO
		%%%%%% INICIO RESPUESTA
\rowcolor[HTML]{FFB183}
		%%%%%%%%%%INICIO TITULO
\multicolumn{3}{|c|}{\cellcolor[HTML]{FFB183}\textbf{6. Respuesta}}   \\ \hline
		%%%%%%%%%% FIN TITULO
		%%%%%%%%%% INICIO RESPUESTA MATEMÁTICA
\multicolumn{3}{|c|}{a) Com bancaria = 174.000 COP} \\
\multicolumn{3}{|c|}{B) $P_{c}$ = 4.858.285 COP + 9.920 COP  = 4.868.205 COP} 
\\ \hline
		
		
		%%%%%%%%%% FIN MATEMÁTICAS
		%%%%%% FIN RESPUESTA
	\end{longtable}
	%Se crean dos lineas en blanco para que no quede el siguiente texto tan pegado
	%\newline \newline %USARLO SI CREES QUE ES NECESARIO
\end{center}
%%%%%%%%%%%%%%%%%%%%%%%%%%FIN 